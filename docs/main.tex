\documentclass[12pt,twoside]{report}
\usepackage{geometry}
\geometry{
    a4paper,
    total={170mm,257mm},
    left=20mm,
    top=20mm,
    }
\usepackage[utf8]{inputenc}
\usepackage[vietnamese,english]{babel}
\usepackage{placeins}
\usepackage{listings}  
\usepackage{pslatex} 
\usepackage{graphicx}
\usepackage{multirow}
\usepackage{fancyhdr}
\usepackage{pdfpages}
\graphicspath{{images/}{../images/}}

\title{
    {Thiết kế và triển khai hệ thống phát hiện xâm nhập dựa trên kĩ thuật máy học}\\
    {\includegraphics[width=4cm]{logo}}
}
\author{
    Nguyễn Đức Thông
    \\
    Phạm Ngọc Hiếu Minh
    \\
    Trần Thanh Tín
}

\pagestyle{fancy}
\fancyhf{}
\rhead{\thepage}
\lfoot{GVHD: TS. Đàm Quang Hồng Hải}
\rfoot{SVTH: 14520901 – 14520533 - 14520960}

\begin{document}
\selectlanguage{vietnamese}
\pagenumbering{gobble}
\includepdf[pages={1}]{cover.pdf}
\newpage
\begin{center}
	{\LARGE \textbf{Lời nói đầu}}
\end{center}
Bên cạnh sự phát triển nhanh chóng và những khả năng mạnh mẽ của công nghệ thông tin thì những vấn đề nhạy cảm như
an toàn thông tin luôn làm cho chúng ta cảm thấy dè dặt. Chúng ta cần phải tăng cường khả năng an toàn thông tin để tránh
khỏi sự mất mát dữ liệu do các lỗ hỏng bảo mật hay bị hacker, virus tấn công. Hàng loạt các giải pháp, cơ chế được các
chuyên gia đưa ra để ngăn chặn, giảm thiểu rủi ro mà các lỗ hỏng bảo mật mang lại. Một trong những giải pháp có thể
đáp ứng tốt cho vấn đề này là Triển khai hệ thống phát hiện xâm nhập dựa trên kĩ thuật máy học.
\par
Hệ thống phát hiện xâm nhập dựa trên kĩ thuật máy học là hệ thống có khả năng giám sát các luồng dữ liệu lưu thông trên mạng
có khả năng phát hiện những hành động khả nghi, những xâm nhập trái phép cũng như các dấu hiệu khai thác bất hợp pháp
nguồn tài nguyên của hệ thống mà từ đó ảnh hưởng tính ổn định, toàn vẹn và sẵn sàng của hệ thống mạng.
Khi triển khai một hệ thống sử dụng máy học, thường phải giải quyết hai bài toán, một là chi phí triển khai,
hai là khả năng đáp ứng linh hoạt của nó trước sự phát triển nhanh chóng của thời đại công nghệ hiện nay.
Máy học ngày càng được tin dùng như 1 giải pháp tối ưu có thể đáp ứng cả hai yêu cầu này.
\newpage
\begin{center}
	{\LARGE \textbf{Lời cảm ơn}}
\end{center}
\par
Chúng em xin chân thành cảm ơn Thầy Đàm Quang Hồng Hải đã tận tình hướng dẫn, chỉ bảo chúng em trong suốt thời gian thực hiện đồ án. Cảm ơn thầy vì tất cả những chỉ bảo và tài liệu mà thầy đã cung cấp cho chúng em để có thể hoàn thành được đề tài. Trong quá trình làm việc với Thầy, chúng em không những học hỏi được nhiều kiến thức bổ ích mà còn học được tinh thần làm việc,sự chuyên nghiệp trong công tác giảng dạy và kinh nghiệm nghiên cứu khoa học của Thầy.
\par
Mặc dù chúng em đã cố gắng hoàn thiện đồ án với tất cả sự nỗ lực của bản thân nhưng chắc chắn không thể tránh khỏi những thiếu sót. Kính mong quý Thầy Cô tận tình giúp chúng em hoàn thiện đồ án tốt hơn
\par
Một lần nữa, chúng em xin chân thành cảm ơn và luôn mong nhận được sự đóng góp quý báu của quý Thầy Cô.

\hfill
\begin{minipage}[t]{0.48\textwidth}
	\begin{center}
		TP.HCM, ngày 14 tháng 01 năm 2018\\
		Sinh viên\\
		Nguyễn Đức Thông\\
		Phạm Ngọc Hiếu Minh\\
		Trần Thanh Tín\\
	\end{center}
\end{minipage}
\newpage
\begin{center}
	{\LARGE \textbf{Lời cam đoan}}
\end{center}
\par
Nhóm em, Nguyễn Đức Thông, Phạm Ngọc Hiếu Minh, Trần Thanh Tín xin cam đoan tất cả các nội dung và tài liệu trình bày trong đồ án này là thành quả của việc tự nghiên cứu, tổng hợp các kiến thức đã được học và làm việc thực tế của nhóm. Các thông tin trích dẫn từ các nguồn tài liệu khác nhau đều được ghi rõ và kèm với phần tài liệu tham khảo
\par
Chúng em xác nhận đồ án này là sản phẩm của nhóm chúng em thực hiện dưới sự hướng dẫn của TS. Đàm Quang Hồng Hải cùng với những sự giúp đỡ khác đều đã được ghi rõ trong báo cáo này.

\hfill
\begin{minipage}[t]{0.48\textwidth}
	\begin{center}
		TP.HCM, ngày 14 tháng 01 năm 2018\\
		Sinh viên\\
		Nguyễn Đức Thông\\
		Phạm Ngọc Hiếu Minh\\
		Trần Thanh Tín\\
	\end{center}
\end{minipage}
\newpage
\tableofcontents
\listoffigures
\listoftables
\newpage
\pagenumbering{arabic}
\chapter{Giới thiệu}
\subsection{Đề tài}
\subsection{Đặt vấn đề}
\subsection{Mục tiêu}
\newpage
\chapter{Lí thuyết}
\section{Hệ thống phát hiện xâm nhập}
\subsubsection{Giới thiệu}
\subsubsection{HIDS}
\subsubsection{NIDS}
\subsubsection{Giới hạn}
\section{Khai phá dữ liệu}
\subsection{Khái niệm}
Là quá trình tính toán để tìm ra các mẫu trong các bộ dữ liệu lớn liên quan đến các phương pháp tại giao điểm của máy học, thống kê và các hệ thống cơ sở dữ liệu. 
Đây là một lĩnh vực liên ngành của khoa học máy tính. Mục tiêu tổng thể của quá trình khai thác dữ liệu là trích xuất thông tin từ một bộ dữ liệu và chuyển nó thành một cấu trúc dễ hiểu để sử dụng tiếp. 
Ngoài bước phân tích thô, nó còn liên quan tới cơ sở dữ liệu và các khía cạnh quản lý dữ liệu, xử lý dữ liệu trước, suy xét mô hình và suy luận thống kê, các thước đo thú vị, các cân nhắc phức tạp, xuất kết quả về các cấu trúc được phát hiện, hiện hình hóa và cập nhật trực tuyến.
\subsection{Các bước trong quá trình khai phá}
Quá trình được thực hiện qua 9 bước:
\begin{enumerate}
    \item \textbf{Tìm hiểu lĩnh vực của bài toán (ứng dụng)}: Các mục đích của bài toán,
    các tri thức cụ thể của lĩnh vực.
    \item \textbf{Tạo nên (thu thập) một tập dữ liệu phù hợp.}
    \item \textbf{Làm sạch và tiền xử lý dữ liệu.}
    \item \textbf{Giảm kích thức của dữ liệu, chuyển đổi dữ liệu}: Xác định thuộc tính quan
    trọng, giảm số chiều (số thuộc tính), biểu diễn bất biến.
    \item \textbf{Lựa chọn chức năng khai phá dữ liệu}: Phân loại, gom cụm, dự báo, sinh
    ra các luật kết hợp.
    \item \textbf{Lựa chọn/ Phát triển (các) giải thuật khai phá dữ liệu phù hợp.}
    \item \textbf{Tiến hành khai phá dữ liệu.}
    \item \textbf{Đánh giá mẫu thu được và biểu diễn tri thức}: Hiển thị hóa, chuyển đổi, bỏ
    đi các mẫu dư thừa,…
    \item \textbf{Sử dụng tri thức được khai phá.}
\end{enumerate}
Quá trình khám phá tri thức theo cách nhìn của giới nghiên cứu về các hệ
thống dữ liệu và kho dữ liệu về quá trình khám phá tri thức
\begin{itemize}
    \item Chuẩn bị dữ liệu \textit{(data preparation)}, bao gồm các quá trình làm sạch dữ liệu
    \textit{(data cleaning)}, tích hợp dữ liệu \textit{(data integration)}, chọn dữ liệu \textit{(data selection)},
    biến đổi dữ liệu \textit{(data transformation)}.
    \item Khai thác dữ liệu \textit{(data mining)}: xác định nhiệm vụ khai thác dữ liệu và lựa
    chọn kỹ thuật khai thác dữ liệu. Kết quả cho ta một nguồn tri thức thô.
    \item Đánh giá \textit{(evaluation)}: dựa trên một số tiêu chí tiến hành kiểm tra và lọc
    nguồn tri thức thu được. 
    \item Triển khai \textit{(deployment)}.
    \item Quá trình khai thác tri thức không chỉ là một quá trình tuần tự từ bước đầu
    tiên đến bước cuối cùng mà là một quá trình lặp và có quay trở lại các bước đã qua.
\end{itemize}
\begin{figure}[!htbp]
    \centering
    \includegraphics[scale=0.7]{data_mining}
    \caption{Quá trình khai phá tri thức}
    \label{fig:x cubed graph}
\end{figure}
\FloatBarrier
\subsection{Ứng dụng của khai phá dữ liệu}
\begin{itemize}
    \item Kinh tế - ứng dụng trong kinh doanh, tài chính, tiếp thị bán hàng, bảo hiểm,
    thương mại, ngân hàng, … Đưa ra các bản báo cáo giàu thông tin; phân tích rủi ro
    trước khi đưa ra các chiến lược kinh doanh, sản xuất; phân loại khách hàng từ đó phân định thị trường, thị phần; …
    \item Khoa học: Thiên văn học – dự đoán đường đi các thiên thể, hành tinh, …;
    Công nghệ sinh học – tìm ra các gen mới, cây con giống mới, …; …
    \item Web: các công cụ tìm kiếm.  
\end{itemize}
\section{Máy học}
\subsection{Giới thiệu}
Những năm gần đây, AI - Artificial Intelligence (Trí Tuệ Nhân Tạo), và cụ thể hơn là Machine Learning (Học Máy hoặc Máy Học) nổi lên như một bằng chứng của cuộc cách mạng công nghiệp lần thứ tư (1 - động cơ hơi nước, 2 - năng lượng điện, 3 - công nghệ thông tin). 
Trí Tuệ Nhân Tạo đang len lỏi vào mọi lĩnh vực trong đời sống mà có thể chúng ta không nhận ra. 
Xe tự hành của Google và Tesla, hệ thống tự tag khuôn mặt trong ảnh của Facebook, trợ lý ảo Siri của Apple, hệ thống gợi ý sản phẩm của Amazon, hệ thống gợi ý phim của Netflix, máy chơi cờ vây AlphaGo của Google DeepMind, …, chỉ là một vài trong vô vàn những ứng dụng của AI/Machine Learning.
\par
Machine Learning là một tập con của AI. 
Theo định nghĩa của Wikipedia, Machine learning is the subfield of computer science that “gives computers the ability to learn without being explicitly programmed”. 
Nói đơn giản, Machine Learning là một lĩnh vực nhỏ của Khoa Học Máy Tính, nó có khả năng tự học hỏi dựa trên dữ liệu đưa vào mà không cần phải được lập trình cụ thể.
\par
Những năm gần đây, khi mà khả năng tính toán của các máy tính được nâng lên một tầm cao mới và lượng dữ liệu khổng lồ được thu thập bởi các hãng công nghệ lớn, Machine Learning đã tiến thêm một bước dài và một lĩnh vực mới được ra đời gọi là Deep Learning (Học Sâu - thực sự tôi không muốn dịch từ này ra tiếng Việt). 
Deep Learning đã giúp máy tính thực thi những việc tưởng chừng như không thể vào 10 năm trước: phân loại cả ngàn vật thể khác nhau trong các bức ảnh, tự tạo chú thích cho ảnh, bắt chước giọng nói và chữ viết của con người, giao tiếp với con người, hay thậm chí cả sáng tác văn hay âm nhạc
\subsection{Phân nhóm các thuật toán}
Theo phương thức học, các thuật toán Machine Learning thường được chia làm 4 nhóm:
\begin{itemize}
\item Supervise learning (học có giám sát)
\item Unsupervised learning ( học không có giám sát)
\item Semi-supervised lerning ( Học bán giám sát)
\item Reinforcement learning ( Học củng cố)
\end{itemize}
\begin{enumerate}
\item \textbf{Học có giám sát (Supervised Learning)}
\par
Supervised learning là thuật toán dự đoán đầu ra (outcome) của một dữ liệu mới (new input) dựa trên các cặp (input, outcome) đã biết từ trước. 
Cặp dữ liệu này còn được gọi là (data, label), tức (dữ liệu, nhãn). 
Supervised learning là nhóm phổ biến nhất trong các thuật toán Machine Learning
\begin{enumerate}
\item \textbf{Classification (Phân loại)}: Một bài toán được gọi là classification nếu các label của input data được chia thành một số hữu hạn nhóm. 
Ví dụ: Gmail xác định xem một email có phải là spam hay không; các hãng tín dụng xác định xem một khách hàng có khả năng thanh toán nợ hay không. 
Ba ví dụ phía trên được chia vào loại này
\par
\item \textbf{Regression (Hồi quy)}
Nếu label không được chia thành các nhóm mà là một giá trị thực cụ thể. Ví dụ: một căn nhà rộng x m2x m2, có yy phòng ngủ và cách trung tâm thành phố z kmz km sẽ có giá là bao nhiêu?
\par
Gần đây Microsoft có một ứng dụng dự đoán giới tính và tuổi dựa trên khuôn mặt. 
Phần dự đoán giới tính có thể coi là thuật toán Classification, phần dự đoán tuổi có thể coi là thuật toán Regression. 
Chú ý rằng phần dự đoán tuổi cũng có thể coi là Classification nếu ta coi tuổi là một số nguyên dương không lớn hơn 150, chúng ta sẽ có 150 class (lớp) khác nhau
\end{enumerate}
\item \textbf{Học không giám sát ( Unsupervise learning)}
\par
Trong thuật toán này, chúng ta không biết được outcome hay nhãn mà chỉ có dữ liệu đầu vào. 
Thuật toán unsupervised learning sẽ dựa vào cấu trúc của dữ liệu để thực hiện một công việc nào đó, ví dụ như phân nhóm (clustering) hoặc giảm số chiều của dữ liệu (dimension reduction) để thuận tiện trong việc lưu trữ và tính toán.
\par
Một cách toán học, Unsupervised learning là khi chúng ta chỉ có dữ liệu vào XX mà không biết nhãn YYtương ứng.
Những thuật toán loại này được gọi là Unsupervised learning vì không giống như Supervised learning, chúng ta không biết câu trả lời chính xác cho mỗi dữ liệu đầu vào. 
Giống như khi ta học, không có thầy cô giáo nào chỉ cho ta biết đó là chữ A hay chữ B. 
Cụm không giám sát được đặt tên theo nghĩa này.
\par
Các bài toán Unsupervised learning được tiếp tục chia nhỏ thành hai loại:
\begin{enumerate}
\item \textbf{Clustering (phân nhóm)}
\par
Một bài toán phân nhóm toàn bộ dữ liệu XX thành các nhóm nhỏ dựa trên sự liên quan giữa các dữ liệu trong mỗi nhóm. 
Ví dụ: phân nhóm khách hàng dựa trên hành vi mua hàng. Điều này cũng giống như việc ta đưa cho một đứa trẻ rất nhiều mảnh ghép với các hình thù và màu sắc khác nhau, ví dụ tam giác, vuông, tròn với màu xanh và đỏ, sau đó yêu cầu trẻ phân chúng thành từng nhóm. 
Mặc dù không cho trẻ biết mảnh nào tương ứng với hình nào hoặc màu nào, nhiều khả năng chúng vẫn có thể phân loại các mảnh ghép theo màu hoặc hình dạng.
\newline
\newline
\item \textbf{Association (kết hợp)}
\par
Là bài toán khi chúng ta muốn khám phá ra một quy luật dựa trên nhiều dữ liệu cho trước. 
Ví dụ: những khách hàng nam mua quần áo thường có xu hướng mua thêm đồng hồ hoặc thắt lưng; những khán giả xem phim Spider Man thường có xu hướng xem thêm phim Bat Man, dựa vào đó tạo ra một hệ thống gợi ý khách hàng (Recommendation System), thúc đẩy nhu cầu mua sắm.
\end{enumerate}
\item \textbf{Học bán giám sát ( Semi-Supervise Learning)}
\par
Các bài toán khi chúng ta có một lượng lớn dữ liệu XX nhưng chỉ một phần trong chúng được gán nhãn được gọi là Semi-Supervised Learning. Những bài toán thuộc nhóm này nằm giữa hai nhóm được nêu bên trên.
\par
Một ví dụ điển hình của nhóm này là chỉ có một phần ảnh hoặc văn bản được gán nhãn (ví dụ bức ảnh về người, động vật hoặc các văn bản khoa học, chính trị) và phần lớn các bức ảnh/văn bản khác chưa được gán nhãn được thu thập từ internet. 
Thực tế cho thấy rất nhiều các bài toán Machine Learning thuộc vào nhóm này vì việc thu thập dữ liệu có nhãn tốn rất nhiều thời gian và có chi phí cao. 
Rất nhiều loại dữ liệu thậm chí cần phải có chuyên gia mới gán nhãn được (ảnh y học chẳng hạn). 
Ngược lại, dữ liệu chưa có nhãn có thể được thu thập với chi phí thấp từ internet.
\newline
\newline   
\item \textbf{Học củng cố ( Reinforcement Learning)}
\par
Reinforcement learning là các bài toán giúp cho một hệ thống tự động xác định hành vi dựa trên hoàn cảnh để đạt được lợi ích cao nhất (maximizing the performance). Hiện tại, Reinforcement learning chủ yếu được áp dụng vào Lý Thuyết Trò Chơi (Game Theory), các thuật toán cần xác định nước đi tiếp theo để đạt được điểm số cao nhất
\begin{figure}[!htbp]
    \centering
    \includegraphics[scale=0.5]{reinforcement_learning}
    \caption{AlphaGo chơi cờ vây với Lee Sedol. AlphaGo là một ví dụ điển hình của Reinforcement Learing}
    \label{fig:x cubed graph}
\end{figure}
\FloatBarrier
\end{enumerate}
\newpage
\chapter{Công nghệ tiếp cận}
\section{Snort 3}
\subsubsection{Giới thiệu}
\subsubsection{Kiến trúc}
\subsubsection{Cấu hình}
\subsubsection{Mở rộng tính năng}
\section{Scikit}
\subsection{Giới thiệu}
\textbf{Scikit-learn} (viết tắt là \textbf{sklearn}) là một thư viện mã nguồn mở trong ngành machine learning, 
rất mạnh mẽ và thông dụng với cộng đồng Python, được thiết kế trên nền NumPy và SciPy. 
Scikit-learn chứa hầu hết các thuật toán machine learning hiện đại nhất, 
đi kèm với comprehensive documentations. Điểm mạnh của thư viện này là nó được sử dụng phổ biến trong giáo dục và công nghiệp, 
do đó nó luôn được cập nhật và có một cộng đồng người dùng đông đảo.
\begin{figure}[!htbp]
    \centering
    \includegraphics[scale=0.5]{scikit}
    \caption{Scikit}
    \label{fig:x cubed graph}
\end{figure}
\FloatBarrier
\subsection{Tại sao nên dùng Scikit?}
Hiện nay có nhiều thư viện mã nguồn mở phục vụ cho nghiên cứu machine learning. Bên cạnh Scikit-learn, có 2 thư viện nổi bật khác là
\begin{itemize}
    \item LibSVM: Được viết trên C bởi Chih-Chung Chang và Chih-Jen Lin. Như tên gọi của nó, thư viên này chứa các thuật toán SVM (Support Vector Machine), nhóm thuật toán mạnh mẽ hỗ trợ cả regression và classification tasks.
    \item TensorFlows: Do các nhà khoa học của viện nghiên cứu Google Brain phát triển. TensorFlows được viết trên Python và là thư viện mở.
\end{itemize}
Trong khi TensorFlows có vẻ low-level hơn thì Scikit-learn cho phép ta sử dụng ngay các thuật toán quan trọng một cách đơn giản và hiệu quả. 
Nói vậy không có nghĩa Scikit-learn là một thư viện “nông cạn”, Scikit-learn là nền tảng để xây dựng các ML implementations khác (Nilearn, Pylearn2,…). 
Scikit-learn còn là một trong những lựa chọn hàng đầu của các nhà nghiên cứu và phát triển. Đứng sau Scikit-learn là các viện nghiên cứu hàng đầu thế giới, gồm có Inria, Télécom Paristech, Paris-Saclay (Pháp), NYU Moore-Sloan Data Science Environment và Columbia University.
\section{Thuật toán KMeans}
\subsection{Giới thiệu}
Thuật toán \textbf{K-means clustering} \cite{kmeans} do MacQueen giới thiệu trong tài liệu “J. Some Methods for Classification and Analysis of Multivariate Observations” năm 1967.
K-means Clustering là một thuật toán dùng trong các bài toán phân loại/nhóm n đối tượng thành k nhóm dựa trên đặc tính/thuộc tính của đối tượng (k £n nguyên, dương).
Về nguyên lý, có n đối tượng, mỗi đối tượng có m thuộc tính, ta phân chia được các đối tượng thành k nhóm dựa trên các thuộc tính của đối tượng bằng việc áp dụng thuật toán này.
Coi mỗi thuộc tính của đối tượng (đối tượng có m thuộc tính) như một toạ độ của không gian m chiều và biểu diễn đối tượng như một điểm của không gian m chiều. 
\par
Trong thuật toán K-means clustering, chúng ta không biết nhãn (label) của từng điểm dữ liệu. 
Mục đích là làm thể nào để phân dữ liệu thành các cụm (cluster) khác nhau sao cho dữ liệu trong cùng một cụm có tính chất giống nhau.
\par
\textbf{Ví dụ}: Một công ty muốn tạo ra những chính sách ưu đãi cho những nhóm khách hàng khác nhau dựa trên sự tương tác giữa mỗi khách hàng với công ty đó (số năm là khách hàng; số tiền khách hàng đã chi trả cho công ty; độ tuổi; giới tính; thành phố; nghề nghiệp; …). 
Giả sử công ty đó có rất nhiều dữ liệu của rất nhiều khách hàng nhưng chưa có cách nào chia toàn bộ khách hàng đó thành một số nhóm/cụm khác nhau. 
Nếu một người biết Machine Learning được đặt câu hỏi này, phương pháp đầu tiên ta nghĩ đến sẽ là K-means Clustering. 
Sau khi đã phân ra được từng nhóm, nhân viên công ty đó có thể lựa chọn ra một vài khách hàng trong mỗi nhóm để quyết định xem mỗi nhóm tương ứng với nhóm khách hàng nào. 
Phần việc cuối cùng này cần sự can thiệp của con người, nhưng lượng công việc đã được rút gọn đi rất nhiều.
\par
Ý tưởng đơn giản nhất về cluster (cụm) là tập hợp các điểm ở gần nhau trong một không gian nào đó (không gian này có thể có rất nhiều chiều trong trường hợp thông tin về một điểm dữ liệu là rất lớn). 
Hình bên dưới là một ví dụ về 3 cụm dữ liệu.
\begin{figure}[!htbp]
    \centering
    \includegraphics[scale=0.5]{kmeans}
    \caption{Bài toán với 3 clusters.}
    \label{fig:x cubed graph}
\end{figure}
\FloatBarrier
Giả sử mỗi cluster có một điểm đại diện (center) màu vàng. Và những điểm xung quanh mỗi center thuộc vào cùng nhóm với center đó. Một cách đơn giản nhất, xét một điểm bất kỳ, ta xét xem điểm đó gần với center nào nhất thì nó thuộc về cùng nhóm với center đó. Tới đây, chúng ta có một bài toán thú vị: \textit{Trên một vùng biển hình vuông lớn có ba đảo hình vuông, tam giác, và tròn màu vàng như hình trên. Một điểm trên biển được gọi là thuộc lãnh hải của một đảo nếu nó nằm gần đảo này hơn so với hai đảo kia . Hãy xác định ranh giới lãnh hải của các đảo}
\subsection{Phân tích chi tiết}
Đầu tiên là chuẩn bị dữ liệu cần phân cụm. Tiếp theo quyết định số lượng cụm (cluster) cần phân chia. Ở ví dụ này thử chọn số cluster là 3. Ở đây data được thể hiện dưới dạng các điểm cho dễ quan sát. Cự ly của các dữ liệu được hiểu là độ dài đoạn thẳng nối giữa 2 điểm với nhau.
\begin{figure}[!htbp]
    \centering
    \includegraphics[scale=0.5]{kmeans_detail1}
    \caption{Cụm ban đầu.}
    \label{fig:x cubed graph}
\end{figure}
\FloatBarrier
\textbf{Bước 2}
\par
Chọn ngẫu nhiên 3 điểm làm điểm trung tâm của cluster.
\begin{figure}[!htbp]
    \centering
    \includegraphics[scale=0.5]{kmeans_detail2}
    \caption{Chọn ngẫu nhiên trung điểm.}
    \label{fig:x cubed graph}
\end{figure}
\FloatBarrier
\textbf{Bước 3}
\par
Với các điểm dữ liệu không được chọn là điểm trung tâm thì tính toán khoảng cách từ chính điểm đó đến các cluster và quyết định cluster nào gần với mình nhất.
\begin{figure}[!htbp]
    \centering
    \includegraphics[scale=0.5]{kmeans_detail3}
    \caption{Tính toán khoảng cách tới điểm.}
    \label{fig:x cubed graph}
\end{figure}
\FloatBarrier
\textbf{Bước 4}
\par
Từ bước tính toán trên, tiến hành phân loại các điểm về các cluster đã quyết định(cluster gần nó nhất). Vậy là đã phân ra được 3 cụm.
\begin{figure}[!htbp]
    \centering
    \includegraphics[scale=0.5]{kmeans_detail4}
    \caption{Phân thành 3 clusters.}
    \label{fig:x cubed graph}
\end{figure}
\FloatBarrier
\textbf{Bước 5}
\par
Bước trên chúng ta đã thu được 3 cụm, bây giờ tiến hành tính trọng tâm của các điểm dữ liệu của từng cụm. Sau đó di chuyển điểm trung tâm của cụm sang vị trí vừa tính được.
\begin{figure}[!htbp]
    \centering
    \includegraphics[scale=0.5]{kmeans_detail5}
    \caption{Tính trọng điểm.}
    \label{fig:x cubed graph}
\end{figure}
\FloatBarrier
Vị trí mà 3 điểm trung tâm của cluster vừa di chuyển đến được hiểu ngắn gọn chính là điểm trung tâm đang di chuyển đến vị trí chính xác hơn.
\newline
\textbf{Bước 6}
\par
Một lần nữa tiến hành bước 3, tính toán lại khoảng các các điểm đến các điểm trung tâm. Sau đó phân loại lại các điểm dữ liệu về các cụm.
\begin{figure}[!htbp]
    \centering
    \includegraphics[scale=0.5]{kmeans_detail6}
    \caption{Tính toán lại khoảng cách.}
    \label{fig:x cubed graph}
\end{figure}
\FloatBarrier
\textbf{Bước 7}
\par
Sau đó lặp lại quá trình di chuyển cluster trung tâm và phân loại lại các điểm về các cụm gần nhất.
Quá trình này sẽ dừng khi sau khi dữ liệu sau khi phân cụm lại không thay đổi gì so với lần trước.
\begin{figure}[!htbp]
    \centering
    \includegraphics[scale=0.5]{kmeans_detail7}
    \caption{Kết quả.}
    \label{fig:x cubed graph}
\end{figure}
\FloatBarrier
\subsection{Lưu ý}
Trước khi sử dụng phương pháp này, chúng ta phải quyết định trước số lượng cluster, tuy nhiên trong quá trình tính toán số lượng cluster có thể khác với số lượng cluster mình dự đoán nên kết quả sẽ không chính xác.
\par
Vì vậy để giải quyết vấn đề này, để có thể chọn ra số lượng cluster thích hợp thì cần phải phân tích dữ liệu cẩn thận, chạy thử k-means với nhiều biến số số lượng cluster.
\par
Cùng ví dụ trên nếu thay số lượng cluster thành 2, kết quả phân loại sẽ thành ra như sau:
\begin{figure}[!htbp]
    \centering
    \includegraphics[scale=0.5]{kmeans_2_clusters}
    \caption{Khi chia thành 2 cụm.}
    \label{fig:x cubed graph}
\end{figure}
\FloatBarrier
\section{Tập dữ liệu NSL-KDD}
\subsubsection{Giới thiệu}
\subsubsection{Đặc tính}
\section{Flatbuffer}
FlatBuffers là một thư viện serialization đa nền tảng hiệu quả cho C++, C\#, C, Go, Java, JavaScript, TypeScript, PHP và Python. 
Ban đầu nó được tạo ra tại Google để phát triển trò chơi và các ứng dụng quan trọng khác về hiệu suất.
\subsection{Tại sao sử dụng FlatBuffers?}
\begin{itemize}
\item \textbf{Truy cập vào dữ liệu được tuần tự hóa mà không cần phân tích cú pháp / giải nén} - Những gì đặt FlatBuffers ngoài là nó đại diện cho dữ liệu phân cấp trong một bộ đệm nhị phân phẳng theo cách mà nó vẫn có thể được truy cập trực tiếp mà không cần phân tích cú pháp / giải nén, trong khi vẫn hỗ trợ tiến hóa cấu trúc dữ liệu (chuyển tiếp / khả năng tương thích ngược).
\item \textbf{Tốc độ và bộ nhớ hiệu quả} - Bộ nhớ duy nhất cần để truy cập dữ liệu của bạn là bộ đệm. Nó yêu cầu 0 phân bổ bổ sung (bằng C ++, các ngôn ngữ khác có thể thay đổi). FlatBuffers cũng rất thích hợp để sử dụng với mmap (hoặc streaming), chỉ yêu cầu một phần của bộ đệm có trong bộ nhớ. Truy cập gần với tốc độ truy cập cấu trúc thô chỉ với một thêm một hướng (một loại vtable) để cho phép phát triển định dạng và các trường tùy chọn. Đó là nhằm vào các dự án mà chi tiêu thời gian và không gian (phân bổ bộ nhớ nhiều) để có thể truy cập hoặc xây dựng dữ liệu tuần tự là không mong muốn, chẳng hạn như trong trò chơi hoặc bất kỳ ứng dụng nhạy cảm hiệu suất nào khác.
\item \textbf{Linh hoạt} - Các trường tùy chọn có nghĩa là bạn không chỉ có khả năng tương thích tốt và tương thích ngược (ngày càng quan trọng đối với các trò chơi tồn tại lâu dài: không phải cập nhật tất cả dữ liệu với mỗi phiên bản mới!). Nó cũng có nghĩa là bạn có nhiều lựa chọn trong dữ liệu nào bạn viết và dữ liệu nào bạn không sử dụng và cách bạn thiết kế cấu trúc dữ liệu.
\item \textbf{Lượng mã nhỏ} - Một lượng nhỏ mã được tạo ra và chỉ một tiêu đề nhỏ duy nhất là sự phụ thuộc tối thiểu, rất dễ tích hợp.
\item \textbf{Kiểu dữ liệu cứng} - Lỗi xảy ra tại thời gian biên dịch thay vì phải viết kiểm tra thời gian chạy lặp đi lặp lại và dễ bị lỗi. Mã có thể được sinh tự động.
\item \textbf{Sử dụng thuận tiện} - Mã C++ được tạo cho phép truy cập và xây dựng mã. Sau đó, có chức năng tùy chọn để phân tích các lược đồ và các biểu diễn văn bản giống JSON khi chạy hiệu quả nếu cần (bộ nhớ nhanh hơn và hiệu quả hơn các trình phân tích cú pháp JSON khác).
\newline 
Mã Java và Go hỗ trợ tái sử dụng đối tượng. C\# có các trình truy cập dựa trên cấu trúc hiệu quả.
\item \textbf{Đa nền tảng và không có phụ thuộc} - mã C ++ sẽ hoạt động với bất kỳ gcc / clang và VS2010 gần đây nào. Đi kèm với các tệp xây dựng cho các thử nghiệm và mẫu (tệp .mk Android và cmake cho tất cả các nền tảng khác).
\end{itemize}
\newpage
\chapter{Triển khai và thử nghiệm}
\section{Mô hình hệ thống}
Hệ thống được chia làm 2 phần:
\begin{enumerate}
    \item Snort Plugin: có nhiệm vụ nhận và phân tích gói tin thành 41 thuộc tính \cite{featureextraction} phù hợp với tập dữ liệu
    NSL-KDD, sau đó mã hóa dùng Flatbuffer và gửi đến máy chủ thông qua http request
    \item Máy chủ dự đoán: có nhiệm vụ huấn luyện mạng dự đoán dùng thuật toán KMeans với tập dữ liệu NSL-KDD \cite{kmeanswithnslkdd},
    đồng thời khởi tạo máy chủ web có endpoint \textbf{/predict} để snort gửi request đến dự đoán
    và trả về kết quả
\end{enumerate}
\begin{figure}[!htbp]
    \centering
    \includegraphics[scale=0.5]{workflow}
    \caption{Mô hình hoạt động}
    \label{fig:x cubed graph}
\end{figure}
\FloatBarrier
\subsection{Snort Plugin}
\begin{figure}[!htbp]
    \centering
    \includegraphics[scale=0.4]{snort1}
    \caption{Đăng ký plugin với snort}
    \label{fig:x cubed graph}
\end{figure}
\FloatBarrier
Khi có gói tin mới đến, snort sẽ gọi đến plugin thông qua hàm eval để xử lý gói tin.
Ở đây ta chỉ chấp nhận các gói tin UDP, TCP và ICMP. Đầu tiên ta cần phân tích thông tin riêng lẻ của từng gói tin, sau đó đưa vào conversation
để phân tích các thông tin liên quan đến thời gian cũng như máy nhận của các packet.
Nếu gói tin bất thường sẽ được gọi đến DetectionEngine của Snort.
\begin{figure}[!htbp]
    \centering
    \includegraphics[scale=0.4]{snort2}
    \caption{Hàm xử lý khi có gói tin mới}
    \label{fig:x cubed graph}
\end{figure}
\FloatBarrier
Phân tích gói tin để lấy các thông tin cơ bản riêng biệt chuẩn bị để dự đoán
\begin{figure}[!htbp]
    \centering
    \includegraphics[scale=0.4]{snort6}
    \caption{Phân tích các thông tin cơ bản của gói tin}
    \label{fig:x cubed graph}
\end{figure}
\FloatBarrier
\begin{figure}[!htbp]
    \centering
    \includegraphics[scale=0.4]{snort7}
    \caption{Phân tích các thông tin cơ bản của gói tin}
    \label{fig:x cubed graph}
\end{figure}
\FloatBarrier
Sau khi trích xuất các thông tin cần thiết từ gói tin, dữ liệu được tách thành 41 features cho dự đoán, 
được mã hóa bằng Flatbuffer. Chỉ phân tích được các thuộc tính từ 1 đến 10 và từ 23 đến 41. 
Các thuộc tính chứa nội dung từ 10 đến 22 không phân tích được.
\begin{figure}[!htbp]
    \centering
    \includegraphics[scale=0.4]{snort3}
    \caption{Mã hóa thông tin gói tin dùng Flatbuffer}
    \label{fig:x cubed graph}
\end{figure}
\FloatBarrier
\begin{figure}[!htbp]
    \centering
    \includegraphics[scale=0.4]{snort4}
    \caption{Mã hóa thông tin gói tin dùng Flatbuffer}
    \label{fig:x cubed graph}
\end{figure}
\FloatBarrier
Sau khi đã phân tích và mã hóa dùng Flatbuffer, chuỗi đã mã hóa được đến máy chủ dự đoán dùng libcurl
\begin{figure}[!htbp]
    \centering
    \includegraphics[scale=0.4]{snort5}
    \caption{Gọi đến máy chủ dự đoán}
    \label{fig:x cubed graph}
\end{figure}
\FloatBarrier
\subsection{Máy chủ dự đoán}
Máy chủ dự đoán có nhiệm vụ huấn luyện mạng dự đoán, khởi tạo endpoint /predict để plugin snort gửi yêu cầu http đến kiểm tra gói tin.
Máy chủ cũng cung cấp trang thống kê đơn giản để người dùng quan sát
\begin{figure}[!htbp]
    \centering
    \includegraphics[scale=0.5]{server1}
    \caption{Máy chủ dự đoán}
    \label{fig:x cubed graph}
\end{figure}
\FloatBarrier
\begin{figure}[!htbp]
    \centering
    \includegraphics[scale=0.4]{server5}
    \caption{Nhập và xử lý tập dữ liệu}
    \label{fig:x cubed graph}
\end{figure}
\FloatBarrier
\begin{figure}[!htbp]
    \centering
    \includegraphics[scale=0.5]{server4}
    \caption{Dùng thuật toán KMeans tạo 4 clusters}
    \label{fig:x cubed graph}
\end{figure}
\FloatBarrier
Endpoint sẽ phân tích request lấy và giải mã Flatbuffer để lấy được cái thông tin gói tin để đưa vào mạng dự đoán
\begin{figure}[!htbp]
    \centering
    \includegraphics[scale=0.6]{server2}
    \caption{Nhận và giải mã Flatbuffer}
    \label{fig:x cubed graph}
\end{figure}
\FloatBarrier
Sau khi đã giải mã thành công, sẽ đưa vào mạng dự đoán Kmeans để dự đoán và trả về kết quả cho Snort plugin
\begin{figure}[!htbp]
    \centering
    \includegraphics[scale=0.6]{server3}
    \caption{Dự đoán gói tin}
    \label{fig:x cubed graph}
\end{figure}
\FloatBarrier
\section{Thử nghiệm và nhận xét}
\begin{figure}[!htbp]
    \centering
    \includegraphics[scale=0.4]{test1}
    \caption{Màn hình thống kê ban đầu}
    \label{fig:x cubed graph}
\end{figure}
\FloatBarrier
\begin{figure}[!htbp]
    \centering
    \includegraphics[scale=0.4]{test2}
    \caption{Khởi tạo Docker Compose để thử nghiệm}
    \label{fig:x cubed graph}
\end{figure}
\FloatBarrier
\begin{figure}[!htbp]
    \centering
    \includegraphics[scale=0.6]{test3}
    \caption{Thử tấn công Ping Flood sử dụng công cụ hping3}
    \label{fig:x cubed graph}
\end{figure}
\FloatBarrier
\begin{figure}[!htbp]
    \centering
    \includegraphics[scale=0.4]{test4}
    \caption{Không phát hiện được tấn công}
    \label{fig:x cubed graph}
\end{figure}
\FloatBarrier
\begin{figure}[!htbp]
    \centering
    \includegraphics[scale=0.6]{test5}
    \caption{Thử tấn công Scan port bằng công cụ nmap}
    \label{fig:x cubed graph}
\end{figure}
\FloatBarrier
\begin{figure}[!htbp]
    \centering
    \includegraphics[scale=0.4]{test6}
    \caption{Không phát hiện được tấn công}
    \label{fig:x cubed graph}
\end{figure}
\FloatBarrier
\newpage
\chapter{Kết luận}
\subsection{Nhận định}
\subsection{Hướng phát triển}

\newpage
\begin{thebibliography}{9}
	\bibitem{nslkdd}
	M. Tavallaee, E. Bagheri, W. Lu, and A. Ghorbani,
	\textit{A Detailed Analysis of the KDD CUP 99 Data Set}.
	Submitted to Second IEEE Symposium on Computational Intelligence for Security and Defense Applications (CISDA), 2009.

	\bibitem{machinelearning}
	L.Dhanabal, Dr. S.P. Shantharajah,
	\textit{A Study on NSL-KDD Dataset for Intrusion Detection System Based on Classification Algorithms}.
	International Journal of Advanced Research in Computer and Com-
	munication Engineering, Vol. 4, Issue 6, June 2015

	\bibitem{system}
	Sandeep Kumar Chandel,
	\textit{Intrusion Detection System
		using K-Means Data Mining
		and Outlier Detection
		Approach}.
	Masaryk University Faculty of Informatics, Bangalore, Spring 2017

	\bibitem{kmeans}
	K-means Clustering
	\\\texttt{https://machinelearningcoban.com/2017/01/01/kmeans/}

	\bibitem{featureextraction}
	KDD99 Feature Extraction
	\\\texttt{https://github.com/AI-IDS/kdd99\_feature\_extractor}

	\bibitem{kmeanswithnslkdd}
	K-means Clustering with NSL-KDD
	\\\texttt{https://github.com/AnomalyIDSBenchmark/KMeansNSL-KDD/}

\end{thebibliography}
\end{document}